\section{Future works/Problems/Solutions}
% FIXME
Even if after 15-20 epochs the train and validation MSE settles around 40 and does not decrease much afterwards, the 
network seems to be learning. Instead, the dataset can be improved since the network is not able to generalise in such 
cases in which the robot is close to the object or within his arms. This is in fact because of the omniscient 
controller that never gets the robot so near the object, and for the network this is an unseen situation for the 
network.
A solution could be to add some runs in the dataset to explore some peculiar cases where the initial position of the 
robot is near the object or the target position.
Another improvement could be to use different shapes for the object, so that the behaviour that the network should 
learn, that is understanding where the shape entrance is, will be clearer.
%
%\subsection{Abbreviations and Acronyms}\label{AA}
%ReLU
%CNN
%Mean Squared Error (MSE)
%
%
%\subsection{Equations}
%
%\begin{equation}
%a+b=\gamma\label{eq}
%\end{equation}
%
%Use ``\eqref{eq}'' in the middle of the sentence or ``Equation \eqref{eq} is . . .'' at the beginning of a sentence. 
%Use (e.g., \verb|\eqref{Eq}|) cross references.
%
%\subsection{Figures and Tables}
%\paragraph{Positioning Figures and Tables} Place figures and tables at the top and 
%bottom of columns. Avoid placing them in the middle of columns. Large 
%figures and tables may span across both columns. Figure captions should be 
%below the figures; table heads should appear above the tables. Insert 
%figures and tables after they are cited in the text. Use the abbreviation 
%``Fig.~\ref{fig}'', even at the beginning of a sentence.
%
