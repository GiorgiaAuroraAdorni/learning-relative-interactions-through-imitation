\section{Experiments}
Using the learned controller, a new dataset has been generated, in which each run is stopped after 20 seconds, (maximum 
of 200 timesteps). 
In the following figures are shown some visualisation explaining the behavior of the robot controlled by the model.

\subsection{Results}
%\begin{figure}[htbp]
%	\centerline{\includegraphics[width=.5\textwidth]{../datasets/learned/images/10-robot-trajectories.pdf}}
%	\caption{Trajectories of ten randomly selected runs.}
%	\label{fig:trajectories-learned}
%\end{figure}
%
%\begin{figure}[htbp]
%	\centerline{\includegraphics[width=.5\textwidth]{../datasets/learned/images/positions-heatmap.pdf}}
%	\caption{Density of samples in each location.}
%	\label{fig:densisy-learned}
%\end{figure}
%
%\begin{figure}[htbp]
%	\centerline{\includegraphics[width=.5\textwidth]{../datasets/learned/images/distances-from-goal.pdf}}
%	\caption{Distance from goal over time.}
%	\label{fig:distance-from-goal-learned}
%\end{figure}
%
%\begin{figure}[htbp]
%	\centerline{\includegraphics[width=.5\textwidth]{../datasets/learned/images/goal-reached.pdf}}
%	\caption{Distribution of the reached goals over time.}
%	\label{fig:goal-reached-learned}
%\end{figure}
%
%\begin{figure}[htbp]
%	\centerline{\includegraphics[width=.5\textwidth]{../datasets/learned/images/initial-final-positions.pdf}}
%	\caption{Initial and final positions of the robot.}
%	\label{fig:initial-final-positions-learned}
%\end{figure}
