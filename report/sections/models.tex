\subsection{Proposed Model}
The generated dataset is then shuffled and split, based on the single run, into 
the train (70\%, 187'000 samples), the validation (15\%) and the test (15\%) 
sets.

Then, a \emph{Convolutional Neural Network (CNN)} is designed. This model takes 
as inputs the sensor distances and colour data obtained from the laser scanner 
(size: $180 \times 4$) and produces as output the left and the right wheel 
target speeds. 

One peculiarity is that we used convolutional layers with circular padding, 
since the laser scanner returns a 360° view of the world around the robot.
Moreover, the \emph{Rectified Linear Unit (ReLU)} activation function is 
applied after every layer, except the last one, in order to add non-linearities 
to the model. 

The training set data are shuffled in each epoch, so the mini-batches (that are 
of size $2^{14}$) are generated independently. 

The model uses \emph{Adam} optimiser with with learning rate $0.001$, while the 
other parameters have their default values. The training is interrupted using 
\emph{early stopping} or after 500 epochs. 

During the various experiments, four different architectures are evaluated:
\begin{itemize}
	\item \emph{Baseline network}: 3 convolutional and 3 fully-connected
	layers (Table \ref{tab: baseline})
	\item Baseline network plus one max pooling layer (Table \ref{tab: maxpool})
	\item Baseline network plus dropout (Table \ref{tab: baseline} + Table 
	\ref{tab: dropout})
	\item Baseline network plus one max pooling layer and dropout (Table 
	\ref{tab: maxpool} + Table \ref{tab: dropout})
\end{itemize} 

\begin{table}[htbp]
	\caption{Architecture of the Baseline Network}
	\begin{center}
		\begin{tabular}{|c|c|c|c|c|}
			\hline
			\textbf{Layer}&\textbf{Channels} &\textbf{Kernel size} &\textbf{Stride} &\textbf{Padding}\\
			\cline{1-5}
			conv1 &  4 $\rightarrow$ 16 & 5 & 2 & 2, circular \\ \hline
			conv2 & 16 $\rightarrow$ 32 & 5 & 2 & 2, circular \\ \hline
			conv3 & 32 $\rightarrow$  			 32 & 5 & 1 & 2, circular \\ \hline
			fc1 &   45 $\times$ 32 $\rightarrow$ 128 &  &  &  \\ \hline
			fc2 &  128 $\rightarrow$ 128 &  &  &  \\ \hline
			fc3 &  128 $\rightarrow$   2 &  &  &  \\ \hline
		\end{tabular}
		\label{tab: baseline}
	\end{center}
\end{table}

\begin{table}[htbp]
	\caption{Architecture of the Network with Max Pooling}
	\begin{center}
		\begin{tabular}{|c|c|c|c|c|}
			\hline
			\textbf{Layer}&\textbf{Channels} &\textbf{Kernel size} &\textbf{Stride} &\textbf{Padding}\\
			\cline{1-5}
			conv1  &   4 $\rightarrow$  \bfseries	32 & 5 & 2 & 2, circular \\ \hline
			conv2  & \bfseries 32 $\rightarrow$  	96 & 5 & 2 & 2, circular \\ \hline
			\bfseries mpool1 & 					   & \bfseries 3	& \bfseries 3 & \bfseries 1, circular \\ 
			\hline			
			conv3  & \bfseries 96 $\rightarrow$  	96 & 5 & 1 & 2, circular \\ \hline
			fc1    & 15 $\times$ 96 $\rightarrow$ 128 &  &  &  \\ \hline
			fc2    & 128 $\rightarrow$ 128 &  &  &  \\ \hline
			fc3    & 128 $\rightarrow$   2 &  &  &  \\ \hline
			%\multicolumn{5}{l}{$^{\mathrm{a}}$Sample of a Table footnote.}
		\end{tabular}
		\label{tab: maxpool}
	\end{center}
\end{table}

\begin{table}[htbp]
	\caption{Architecture of the Network with Dropout}
	\begin{center}
		\begin{tabular}{|c|c|c|c|c|}
			\hline
			\textbf{Layer}&\textbf{Channels} &\textbf{Kernel size} &\textbf{Stride} &\textbf{Padding}\\
			\cline{1-5}
			\multicolumn{5}{|c|}{...} \\ \hline
			fc1 &  1440 $\rightarrow$ 128 &  &  &  \\ \hline
			\bfseries drop1 & \multicolumn{4}{c|}{\bfseries dropout with p = 0.5} \\ \hline
			fc2 &  128 $\rightarrow$ 128 &  &  &  \\ \hline
			\bfseries drop2 & \multicolumn{4}{c|}{\bfseries dropout with p = 0.5} \\ \hline
			fc3 &  128 $\rightarrow$   2 &  &  &  \\ \hline
			%\multicolumn{5}{l}{$^{\mathrm{a}}$Sample of a Table footnote.}
		\end{tabular}
		\label{tab: dropout}
	\end{center}
\end{table}

