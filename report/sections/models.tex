\section{Proposed Model}
The original dataset, that is the set of all the runs, is then shuffled and split, based on the single run, into the 
train, the validation and the test sets. The proportions are 70\% for training and 15\% each for validation and 
testing. In total, the training set is composed of 187000 samples.
The network trained is a CNN,  which takes as inputs the sensor distances and the camera image readings and as output 
produces the left and the right wheel target speeds. 
The input size is 180x4. The model is trained for 313 epochs, until early stopping interrupts it. 
The training set data are shuffled in each epoch, so the mini-batches (that are of size $2^14$) are generated 
independently. 
Adam is used as optimiser with learning rate $0.01$ and the loss function chosen is the Mean Squared Error (MSE). The 
other parameters have their default values. 
In the network the ReLU non-linearity is applied after every layer except the last one. 
The structure is the following:

\begin{table}[htbp]
	\caption{Architecture of the Baseline Network}
	\begin{center}
		\begin{tabular}{|c|c|c|c|c|}
			\hline
			\textbf{Layer}&\textbf{Channels} &\textbf{Kernel size} &\textbf{Stride} &\textbf{Padding}\\
			\cline{1-5}
			conv1 &  4 $\rightarrow$ 16 & 5 & 2 & 2, circular \\ \hline
			conv2 & 16 $\rightarrow$ 32 & 5 & 2 & 2, circular \\ \hline
			conv3 & 32 $\rightarrow$  			 32 & 5 & 1 & 2, circular \\ \hline
			fc1 &   45 $\times$ 32 $\rightarrow$ 128 &  &  &  \\ \hline
			fc2 &  128 $\rightarrow$ 128 &  &  &  \\ \hline
			fc3 &  128 $\rightarrow$   2 &  &  &  \\ \hline
		\end{tabular}
		\label{tab: baseline}
	\end{center}
\end{table}

\begin{table}[htbp]
	\caption{Architecture of the Network with Max Pooling}
	\begin{center}
		\begin{tabular}{|c|c|c|c|c|}
			\hline
			\textbf{Layer}&\textbf{Channels} &\textbf{Kernel size} &\textbf{Stride} &\textbf{Padding}\\
			\cline{1-5}
			conv1  &   4 $\rightarrow$  \bfseries	32 & 5 & 2 & 2, circular \\ \hline
			conv2  & \bfseries 32 $\rightarrow$  	96 & 5 & 2 & 2, circular \\ \hline
			\bfseries mpool1 & 					   & \bfseries 3	& \bfseries 3 & \bfseries 1, circular \\ 
			\hline			
			conv3  & \bfseries 96 $\rightarrow$  	96 & 5 & 1 & 2, circular \\ \hline
			fc1    & 15 $\times$ 96 $\rightarrow$ 128 &  &  &  \\ \hline
			fc2    & 128 $\rightarrow$ 128 &  &  &  \\ \hline
			fc3    & 128 $\rightarrow$   2 &  &  &  \\ \hline
			%\multicolumn{5}{l}{$^{\mathrm{a}}$Sample of a Table footnote.}
		\end{tabular}
		\label{tab: maxpool}
	\end{center}
\end{table}

\begin{table}[htbp]
	\caption{Architecture of the Network with Dropout}
	\begin{center}
		\begin{tabular}{|c|c|c|c|c|}
			\hline
			\textbf{Layer}&\textbf{Channels} &\textbf{Kernel size} &\textbf{Stride} &\textbf{Padding}\\
			\cline{1-5}
			\multicolumn{5}{|c|}{...} \\ \hline
			fc1 &  1440 $\rightarrow$ 128 &  &  &  \\ \hline
			\bfseries drop1 & \multicolumn{4}{c|}{\bfseries dropout with p = 0.5} \\ \hline
			fc2 &  128 $\rightarrow$ 128 &  &  &  \\ \hline
			\bfseries drop2 & \multicolumn{4}{c|}{\bfseries dropout with p = 0.5} \\ \hline
			fc3 &  128 $\rightarrow$   2 &  &  &  \\ \hline
			%\multicolumn{5}{l}{$^{\mathrm{a}}$Sample of a Table footnote.}
		\end{tabular}
		\label{tab: dropout + mpool}
	\end{center}
\end{table}



\begin{table}[htbp]
	\caption{Architecture of the Network for Task 2}
	\begin{center}
		\begin{tabular}{|c|c|c|c|c|}
			\hline
			\textbf{Layer}&\textbf{Channels} &\textbf{Kernel size} &\textbf{Stride} &\textbf{Padding}\\
			\cline{1-5}
			conv1  &  4 $\rightarrow$ 	32 & 5 & 2 & 2, circular \\ \hline
			conv2  & 32 $\rightarrow$  	96 & 5 & 2 & 2, circular \\ \hline
			mpool1 & 					   & 3	& 3 & 1, circular \\ 
			\hline			
			conv3  & 96 $\rightarrow$  	96 & 5 & 1 & 2, circular \\ \hline
			fc1   &  1440 \textbf{+ 3} $\rightarrow$ 128 &  &  &  \\ \hline
			drop1 & \multicolumn{4}{c|}{dropout with p = 0.5} \\ \hline
			fc2   &  128 $\rightarrow$ 128 &  &  &  \\ \hline
			 drop2 & \multicolumn{4}{c|}{ dropout with p = 0.5} \\ \hline
			fc3 &  128 $\rightarrow$   2 &  &  &  \\ \hline
			%\multicolumn{5}{l}{$^{\mathrm{a}}$Sample of a Table footnote.}
		\end{tabular}
		\label{tab: task 2}
	\end{center}
\end{table}


\subsection{Model performance}
%\begin{figure}[htbp]
%	\centerline{\includegraphics[width=.5\textwidth]{../models/net6/images/initial-positions.pdf}}
%	\caption{Initial positions divided by belonging dataset.}
%	\label{fig:initial-positions}
%\end{figure}
%
%\begin{figure}[htbp]
%	\centerline{\includegraphics[width=.5\textwidth]{../models/net6/images/loss.pdf}}
%	\caption{Comparison of the losses among the train and validation sets.}
%	\label{fig:loss}
%\end{figure}
%
%\begin{figure}[htbp]
%	\centerline{\includegraphics[width=.5\textwidth]{../models/net6/images/distribution-target.pdf}}
%	\caption{Comparison of the distributions of groundtruth and prediction of the validation set.}
%	\label{fig:distribution-target}
%\end{figure}
%
%\begin{figure}[htbp]
%	\centerline{\includegraphics[width=.5\textwidth]{../models/net6/images/regression.pdf}}
%	\caption{Comparison of the $R^2$regressor between groundtruth and prediction of the validation set.}
%	\label{fig:regression}
%\end{figure}
