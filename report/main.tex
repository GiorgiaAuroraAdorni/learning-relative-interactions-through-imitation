\documentclass[conference]{IEEEtran}
%\IEEEoverridecommandlockouts
% The preceding line is only needed to identify funding in the first footnote. If that is unneeded, please comment it out.
\usepackage{cite}
\usepackage{amsmath,amssymb,amsfonts}
\usepackage{algorithmic}
\usepackage{graphicx}
\graphicspath{{images/}}
\usepackage{textcomp}
\usepackage{xcolor}
\usepackage{url}
\usepackage[hidelinks]{hyperref}
\def\BibTeX{{\rm B\kern-.05em{\sc i\kern-.025em b}\kern-.08em
    T\kern-.1667em\lower.7ex\hbox{E}\kern-.125emX}}
\begin{document}

\title{Learning Relative Interactions through Imitation \\ \vspace{0.5\baselineskip}
{
	\large {Università della Svizzera Italiana}\\
	{Faculty of Informatics} \\
	Lugano, Switzerland \\
	\textit{Project for the Robotics course 2019--2020}\\
}
%\thanks{Identify applicable funding agency here. If none, delete this.}
}

\author{\IEEEauthorblockN{Giorgia Adorni}
\IEEEauthorblockA{(giorgia.adorni@usi.ch)}
\and
\IEEEauthorblockN{Elia Cereda}
\IEEEauthorblockA{(elia.cereda@usi.ch)}}

\maketitle
\thispagestyle{plain}
\pagestyle{plain}

\begin{abstract}
In this project we trained a neural network to perform specific interactions 
between a robot and objects in the environment, through imitation learning. In 
particular, we tackle the task of moving the robot to a fixed pose with respect 
to a certain object and later extend our method to handle any arbitrary pose 
around this object.

We show that a simple network, with relatively little training data, is able to 
reach very good performance on the fixed-pose task, while more work is needed 
to perform the arbitrary-pose task satisfactorily. We also explore the effect 
of ambiguities in the sensor readings, in particular caused by symmetries in 
the target object, on the behaviour of the learned controller.

\emph{External Resources}---source code~\cite{github}, pitch 
presentation~\cite{pitch} and final presentation~\cite{final-pitch}.

\end{abstract}

%\begin{IEEEkeywords}
%component, formatting, style, styling, insert
%\end{IEEEkeywords}

\section{Introduction}

In robotics, some tasks are relatively easy to perform with complete knowledge 
of the environment, but become more challenging when the environment is only 
partially observable using a robot's sensors. Imitation learning deals with 
this problem by recording the trajectories of an omniscient controller 
performing the desired task, then training a machine learning model to 
replicate them using just the data from the sensors. As such, the machine 
learning model must learn how to extract the relevant information from the data 
it receives, sidestepping the difficulty of implementing the perception part 
manually.

The target platform we choose for our project is the 
marXbot~\cite{bonani2010marxbot}, a research robot originally designed to study 
collective and swarm robotics. The main characteristic making the marXbot 
interesting for this project is its rotating laser scanner, which perceives 
distances and colours of the objects surrounding the robot.

The experiments are run in Enki~\cite{enki}, a high-performance open-source 
simulator for planar robots, which provides collision detection and limited 
physics support for robots evolving on a flat surface. Moreover, it can 
simulate groups of robots hundreds of times faster than real-time.

\begin{figure}[htbp]
	\centerline{\includegraphics[width=.8\columnwidth]{introduction/marxbot}}
	\caption{Actual image and exploded CAD view of a marXbot.}
	\label{fig:marxbot}
\end{figure}

The task that we 

%The main objective of this project is learning to perform specific 
%interactions between the robot and objects in the 
%environment.
%Write an omniscient controller that performs the desired interaction with 
%complete knowledge of the environment (e.g. 
%position the robot at a certain location relative to an object) using Enki.
%Generate a dataset of simulation runs through Enki. 
%Through imitation learning, train an end-to-end neural network that receives 
%as inputs the sensor distances and the 
%camera image readings and produces commands for the motors that are the left 
%and the right wheel target speeds.
%Evaluate the model trained using Enki.
% 

\section{Controller}

\subsection{The omniscient controller}

The robot is moved following an optimal “omniscient” controller \cite{park2011smooth}. This controller has complete 
knowledge of the environment. Using the known poses of the robot and of the target object, the omniscient control moves 
the robot to the goal pose, as fast as possible. In particular, the controller checks if the euclidean distance from 
the goal is less than 1mm and if the robot orientation is less than 0.5 degrees from the goal orientation.


\section{Task 1}
\section{Data Generation Through Enki Simulations}
Using Enki, a dataset containing 2000 simulation runs is generated. Each run differs in the initial position of the 
robot. In particular, each of these sets up a world with:
\begin{itemize}
	\item a horseshoe-shaped object, that represents a hypothetical docking station, always in the pose $(x=0, y=0, 
	\theta=0)$
	\item a marXbot, positioned randomly around the object, up to a maximum distance. The goal pose is in front of the 
	two arms of the object.
\end{itemize}

The dataset records the run as sequence of time-steps containing the following information: 
\begin{itemize}
	\item run
	\item step
	\item name
	\item initial position (axis: x, y) 
	\item initial angle
	\item goal position (axis: x, y)
	\item goal angle
	\item position (axis: x, y) 
	\item angle
	\item wheel target speeds (wheel: l, r)
	\item scanner distances (scanner angle), 
	\item scanner image": (scanner angle, channel: r, g, b), 
	\item goal reached 
	\item goal position distance
	\item goal angle distance
\end{itemize}
The run is stopped if the robot reaches the target or, in any case, after 20 seconds, that means that each run can 
contain a maximum of 200 time-steps.

\subsection{Dataset visualisations}
%\begin{figure}[htbp]
%	\centerline{\includegraphics[width=.5\textwidth]{../datasets/omniscient/images/10-robot-trajectories.pdf}}
%	\caption{Trajectories of ten randomly selected runs.}
%	\label{fig:trajectories-omniscient}
%\end{figure}
%
%\begin{figure}[htbp]
%	\centerline{\includegraphics[width=.5\textwidth]{../datasets/omniscient/images/positions-heatmap.pdf}}
%	\caption{Density of samples in each location.}
%	\label{fig:densisy-omniscient}
%\end{figure}
%
%\begin{figure}[htbp]
%	\centerline{\includegraphics[width=.5\textwidth]{../datasets/omniscient/images/distances-from-goal.pdf}}
%	\caption{Distance from goal over time.}
%	\label{fig:distance-from-goal-omniscient}
%\end{figure}
%
%\begin{figure}[htbp]
%	\centerline{\includegraphics[width=.5\textwidth]{../datasets/omniscient/images/goal-reached.pdf}}
%	\caption{Distribution of the reached goals over time.}
%	\label{fig:goal-reached-omniscient}
%\end{figure}
%
%\begin{figure}[htbp]
%	\centerline{\includegraphics[width=.5\textwidth]{../datasets/omniscient/images/initial-final-positions.pdf}}
%	\caption{Initial and final positions of the robot.}
%	\label{fig:initial-final-positions-omniscient}
%\end{figure}

%Laser scanner readings and control signals over time

\section{Proposed Model}
The original dataset, that is the set of all the runs, is then shuffled and split, based on the single run, into the 
train, the validation and the test sets. The proportions are 70\% for training and 15\% each for validation and 
testing. In total, the training set is composed of 187000 samples.
The network trained is a CNN,  which takes as inputs the sensor distances and the camera image readings and as output 
produces the left and the right wheel target speeds. 
The input size is 180x4. The model is trained for 313 epochs, until early stopping interrupts it. 
The training set data are shuffled in each epoch, so the mini-batches (that are of size $2^14$) are generated 
independently. 
Adam is used as optimiser with learning rate $0.01$ and the loss function chosen is the Mean Squared Error (MSE). The 
other parameters have their default values. 
In the network the ReLU non-linearity is applied after every layer except the last one. 
The structure is the following:

\begin{table}[htbp]
	\caption{Architecture of the Baseline Network}
	\begin{center}
		\begin{tabular}{|c|c|c|c|c|}
			\hline
			\textbf{Layer}&\textbf{Channels} &\textbf{Kernel size} &\textbf{Stride} &\textbf{Padding}\\
			\cline{1-5}
			conv1 &  4 $\rightarrow$ 16 & 5 & 2 & 2, circular \\ \hline
			conv2 & 16 $\rightarrow$ 32 & 5 & 2 & 2, circular \\ \hline
			conv3 & 32 $\rightarrow$  			 32 & 5 & 1 & 2, circular \\ \hline
			fc1 &   45 $\times$ 32 $\rightarrow$ 128 &  &  &  \\ \hline
			fc2 &  128 $\rightarrow$ 128 &  &  &  \\ \hline
			fc3 &  128 $\rightarrow$   2 &  &  &  \\ \hline
		\end{tabular}
		\label{tab: baseline}
	\end{center}
\end{table}

\begin{table}[htbp]
	\caption{Architecture of the Network with Max Pooling}
	\begin{center}
		\begin{tabular}{|c|c|c|c|c|}
			\hline
			\textbf{Layer}&\textbf{Channels} &\textbf{Kernel size} &\textbf{Stride} &\textbf{Padding}\\
			\cline{1-5}
			conv1  &   4 $\rightarrow$  \bfseries	32 & 5 & 2 & 2, circular \\ \hline
			conv2  & \bfseries 32 $\rightarrow$  	96 & 5 & 2 & 2, circular \\ \hline
			\bfseries mpool1 & 					   & \bfseries 3	& \bfseries 3 & \bfseries 1, circular \\ 
			\hline			
			conv3  & \bfseries 96 $\rightarrow$  	96 & 5 & 1 & 2, circular \\ \hline
			fc1    & 15 $\times$ 96 $\rightarrow$ 128 &  &  &  \\ \hline
			fc2    & 128 $\rightarrow$ 128 &  &  &  \\ \hline
			fc3    & 128 $\rightarrow$   2 &  &  &  \\ \hline
			%\multicolumn{5}{l}{$^{\mathrm{a}}$Sample of a Table footnote.}
		\end{tabular}
		\label{tab: maxpool}
	\end{center}
\end{table}

\begin{table}[htbp]
	\caption{Architecture of the Network with Dropout}
	\begin{center}
		\begin{tabular}{|c|c|c|c|c|}
			\hline
			\textbf{Layer}&\textbf{Channels} &\textbf{Kernel size} &\textbf{Stride} &\textbf{Padding}\\
			\cline{1-5}
			\multicolumn{5}{|c|}{...} \\ \hline
			fc1 &  1440 $\rightarrow$ 128 &  &  &  \\ \hline
			\bfseries drop1 & \multicolumn{4}{c|}{\bfseries dropout with p = 0.5} \\ \hline
			fc2 &  128 $\rightarrow$ 128 &  &  &  \\ \hline
			\bfseries drop2 & \multicolumn{4}{c|}{\bfseries dropout with p = 0.5} \\ \hline
			fc3 &  128 $\rightarrow$   2 &  &  &  \\ \hline
			%\multicolumn{5}{l}{$^{\mathrm{a}}$Sample of a Table footnote.}
		\end{tabular}
		\label{tab: dropout + mpool}
	\end{center}
\end{table}



\begin{table}[htbp]
	\caption{Architecture of the Network for Task 2}
	\begin{center}
		\begin{tabular}{|c|c|c|c|c|}
			\hline
			\textbf{Layer}&\textbf{Channels} &\textbf{Kernel size} &\textbf{Stride} &\textbf{Padding}\\
			\cline{1-5}
			conv1  &  4 $\rightarrow$ 	32 & 5 & 2 & 2, circular \\ \hline
			conv2  & 32 $\rightarrow$  	96 & 5 & 2 & 2, circular \\ \hline
			mpool1 & 					   & 3	& 3 & 1, circular \\ 
			\hline			
			conv3  & 96 $\rightarrow$  	96 & 5 & 1 & 2, circular \\ \hline
			fc1   &  1440 \textbf{+ 3} $\rightarrow$ 128 &  &  &  \\ \hline
			drop1 & \multicolumn{4}{c|}{dropout with p = 0.5} \\ \hline
			fc2   &  128 $\rightarrow$ 128 &  &  &  \\ \hline
			 drop2 & \multicolumn{4}{c|}{ dropout with p = 0.5} \\ \hline
			fc3 &  128 $\rightarrow$   2 &  &  &  \\ \hline
			%\multicolumn{5}{l}{$^{\mathrm{a}}$Sample of a Table footnote.}
		\end{tabular}
		\label{tab: task 2}
	\end{center}
\end{table}


\subsection{Model performance}
%\begin{figure}[htbp]
%	\centerline{\includegraphics[width=.5\textwidth]{../models/net6/images/initial-positions.pdf}}
%	\caption{Initial positions divided by belonging dataset.}
%	\label{fig:initial-positions}
%\end{figure}
%
%\begin{figure}[htbp]
%	\centerline{\includegraphics[width=.5\textwidth]{../models/net6/images/loss.pdf}}
%	\caption{Comparison of the losses among the train and validation sets.}
%	\label{fig:loss}
%\end{figure}
%
%\begin{figure}[htbp]
%	\centerline{\includegraphics[width=.5\textwidth]{../models/net6/images/distribution-target.pdf}}
%	\caption{Comparison of the distributions of groundtruth and prediction of the validation set.}
%	\label{fig:distribution-target}
%\end{figure}
%
%\begin{figure}[htbp]
%	\centerline{\includegraphics[width=.5\textwidth]{../models/net6/images/regression.pdf}}
%	\caption{Comparison of the $R^2$regressor between groundtruth and prediction of the validation set.}
%	\label{fig:regression}
%\end{figure}

\subsection{Experiments}
\subsubsection{Experiment 1}
The first experiment performed compares the different architectures using the 
\emph{Mean Squared Error (MSE)} loss function.
The baseline model is able to reach the goal, but with little precision and 
often colliding with the object, as shown in Figure~\ref{fig:baseline}.

\begin{figure}[htbp]
	\centerline{\includegraphics[width=\columnwidth]{experiments/1/demo-trajectories-baseline}}
	\caption{Trajectories of the controller learned from the baseline network.}
	\label{fig:baseline}
\end{figure}

\begin{figure}[htbp]
	\centerline{\includegraphics[width=\columnwidth]{experiments/1/regression-validation-baseline}}
	\caption{$R^2$ regressor on the validation set of the baseline network.}
	\label{fig:regression-baseline}
\end{figure}

Adding either max pooling or dropout alone does not solve the problem, but 
combining them results in a visible improvement: the robot reaches the goal 
position more precisely even if oscillating a bit.

\begin{figure}[htbp]
	\centerline{\includegraphics[width=\columnwidth]{experiments/1/demo-trajectories-maxpool+dropout}}
	\caption{Trajectories of the controller learned from the max pooling + 
	dropout network.}
	\label{fig:maxpool+dropout}
\end{figure}

\begin{figure}[htbp]
	\centerline{\includegraphics[width=\columnwidth]{experiments/1/regression-validation-maxpool+dropout}}
	\caption{$R^2$ regressor on the validation set of the max pooling + dropout 
	network.}
	\label{fig:regression-maxpool+dropout}
\end{figure}

The regression coefficients of the angular velocity, displayed in 
Figure~\ref{fig:regression-baseline} and \ref{fig:regression-maxpool+dropout}, 
increase from $0.54$ to $0.64$, confirming the improvement of the second model.
A shown in Figure~\ref{fig:loss}, it shows also a lower validation loss (in 
red) and does not overfit toward the end of training, like the baseline model 
(in orange).

\begin{figure}[htbp]
\centerline{\includegraphics[width=.8\columnwidth]{experiments/1/loss}}
	\caption{Comparison of the losses among train and validation sets.}
	\label{fig:loss}
\end{figure}

Finally, the end positions are more tightly clustered over the goal.
Both the heatmaps in Figure~\ref{fig:heatmaps} show a tendency to rotate around 
the object, which are caused by its symmetry. We will explore this in 
Section~\ref{experiment3}.

\begin{figure}[htbp]
	\centerline{\includegraphics[width=\columnwidth]{experiments/1/heatmaps}}
	\caption{Positions heatmaps.}
	\label{fig:heatmaps}
\end{figure}

\begin{figure}[htbp]
	\centerline{\includegraphics[width=\columnwidth]{experiments/1/final-positions}}
	\caption{Final positions.}
	\label{fig:final-positions}
\end{figure}

Even though the overall behaviour of this model is good, the main drawback is a 
slight systematic error in the final orientation that we can see in 
Figure~\ref{fig:distance-from-goal-learned}.

\begin{figure}[htbp]
	\centerline{\includegraphics[width=\columnwidth]{experiments/1/distances-from-goal}}
	\caption{Distance from goal over time.}
	\label{fig:distance-from-goal-learned}
\end{figure}

\subsubsection{Experiment 2}
The second experiment performed compares the previous model with one with the 
same network architecture, but that uses a second loss function, \emph{Smooth 
L1} \cite{smoothl1}, which is less sensitive to outliers than \emph{MSE} and in 
some cases prevents exploding gradients. It is computed using \eqref{smoothl1}

\begin{equation}
\text{loss}(x, y) = \frac{1}{n}\sum_{i}z_i
\label{smoothl1}
\end{equation}

where $z_i$ is given by

\begin{equation}
z_i = 
\begin{cases}
0.5 (x_i-y_i)^2, &\text{ if } |x_i-y_i|<1 \\
|x_i-y_i| - 0.5, &\text{ otherwise}
\end{cases}
\end{equation}

 
Although it results in less precise final positions, it solves the oscillation 
issue, as shown in Figure~\ref{fig:demo-trajectories}.

\begin{figure}[htbp]
	\centerline{\includegraphics[width=\columnwidth]{experiments/2/demo-trajectories}}
	\caption{Trajectories.}
	\label{fig:demo-trajectories}
\end{figure}

The regression coefficients of the angular velocity, shown in 
Figure~\ref{fig:regression-validation}, decrease from $0.64$ to $0.58$, 
confirming the superiority of the previous model.

\begin{figure}[htbp]
	\centerline{\includegraphics[width=\columnwidth]{experiments/2/regression-validation}}
	\caption{$R^2$ regressor on the validation set.}
	\label{fig:regression-validation}
\end{figure}

\subsubsection{Experiment 3}
\label{experiment3}

The monochromatic goal object shown so far has symmetries that make the 
trajectory to follow ambiguous, causing the robots converge to the goal in a 
spiral path, as visualised in Figure~\ref{fig:demo-circle-trajectories}.
This issue is solved in this final experiments that consists in removing any 
ambiguity in the sensor readings by using a polychromatic goal object that has 
different colours for each of its faces.

\begin{figure}[htbp]
	\centerline{\includegraphics[width=\columnwidth]{experiments/3/monochromatic-polychromatic}}
	\caption{Comparison of the trajectories of the monochromatic and 
	polychromatic goal object.}
	\label{fig:demo-circle-trajectories}
\end{figure}

The same network architecture and loss function of the first experiment are 
used to train the the model with a polychromatic object, and result in a 
significant improvement both in regression coefficient of the angular 
velocities, shown in Figure~\ref{fig:regression-3}, and in training and 
validation losses (in green and red), in Figure~\ref{fig:loss-3}.

\begin{figure}[htbp]
	\centerline{\includegraphics[width=\columnwidth]{experiments/3/regression-validation}}
	\caption{$R^2$ regressor on the validation set.}
	\label{fig:regression-3}
\end{figure}

\begin{figure}[htbp]
	
	\centerline{\includegraphics[width=.8\columnwidth]{experiments/3/loss}}
	\caption{Comparison of the losses among train and validation sets.}
	\label{fig:loss-3}
\end{figure}

Finally, Figure~\ref{fig:heatmap-final-positions} shows how the end positions 
are more tightly clustered over the goal than before. Moreover, the model is 
able to follow the optimal trajectories without rotating around the object.
Also in terms of convergence, the robot reaches the goal more precisely, as 
shown in Figure~\ref{fig:distance-from-goal-learned3}, and sometimes even 
faster than the omniscient controller.

\begin{figure}[htbp]
	\centerline{\includegraphics[width=\columnwidth]{experiments/3/heatmap-final-positions}}
	\caption{Positions heatmap and final positions.}
	\label{fig:heatmap-final-positions}
\end{figure}

\begin{figure}[htbp]
	\centerline{\includegraphics[width=\columnwidth]{experiments/3/distances-from-goal}}
	\caption{Distance from goal over time.}
	\label{fig:distance-from-goal-learned3}
\end{figure}


\section{Task 2}
As an additional task, we experimented with arbitrary goal poses relative to a 
certain object. 

To this end, we first generated a dataset with the same omniscient controller 
as before, but with random goal poses located in a ring around the object, as 
shown in Figure~\ref{fig:goal-positions}.

\begin{figure}[htbp]
	\centerline{\includegraphics[width=.8\columnwidth]{task2/goal-positions}}
	\caption{Goal positions.}
	\label{fig:goal-positions}
\end{figure}

Then, we implemented a neural network which receives the desired goal pose as 
input to the first fully-connected layer. The architecture used is shown in 
Table~\ref{tab: task 2}.

\begin{table}[htbp]
	\caption{Architecture of the Network for Task 2}
	\begin{center}
		\begin{tabular}{|c|c|c|c|c|}
			\hline
			\textbf{Layer}&\textbf{Channels} &\textbf{Kernel size} 
			&\textbf{Stride} &\textbf{Padding}\\
			\cline{1-5}
			conv1  &  4 $\rightarrow$ 	32 & 5 & 2 & 2, circular \\ \hline
			conv2  & 32 $\rightarrow$  	96 & 5 & 2 & 2, circular \\ \hline
			mpool1 & 					   & 3	& 3 & 1, circular \\ 
			\hline			
			conv3  & 96 $\rightarrow$  	96 & 5 & 1 & 2, circular \\ \hline
			fc1   &  1440 \textbf{+ 3} $\rightarrow$ 128 &  &  &  \\ \hline
			drop1 & \multicolumn{4}{c|}{dropout with p = 0.5} \\ \hline
			fc2   &  128 $\rightarrow$ 128 &  &  &  \\ \hline
			 drop2 & \multicolumn{4}{c|}{ dropout with p = 0.5} \\ \hline
			fc3 &  128 $\rightarrow$   2 &  &  &  \\ \hline
			%\multicolumn{5}{l}{$^{\mathrm{a}}$Sample of a Table footnote.}
		\end{tabular}
		\label{tab: task 2}
	\end{center}
\end{table}

From the plots shown in Figures~\ref{fig:positions-heatmap-2} and 
\ref{fig:goal-reached-2} we can see that the unchanged omniscient controller is 
not perfectly suited for this task: while it was possible to skip collision 
avoidance before, here it results in many runs getting stuck on the object.

\begin{figure}[htbp]
	\centerline{\includegraphics[width=.8\columnwidth]{task2/positions-heatmap}}
	\caption{Positions heatmap.}
	\label{fig:positions-heatmap-2}
\end{figure}

\begin{figure}[htbp]
	\centerline{\includegraphics[width=.8\columnwidth]{task2/goal-reached}}
	\caption{Time to reach the goal.}
	\label{fig:goal-reached-2}
\end{figure}

The distances from goal, in Figure~\ref{fig:distances-from-goal-2}, confirm 
this issue, but in general the performance is good when there are no collisions.

\begin{figure}[htbp]
	\centerline{\includegraphics[width=\columnwidth]{task2/distances-from-goal}}
	\caption{Distance from goal.}
	\label{fig:distances-from-goal-2}
\end{figure}

This results in trajectories like those shown in 
Figure~\ref{fig:demo-circle-trajectories-2}.

\begin{figure}[htbp]
	\centerline{\includegraphics[width=\columnwidth]{task2/demo-circle-trajectories}}
	\caption{Trajectories of the controller learned.}
	\label{fig:demo-circle-trajectories-2}
\end{figure}

\section{Future works}
In task 2, it appears that the performance of the neural network is quite good 
at moving in the general direction of the goal, less so at arriving with the 
correct orientation or stopping at the right time.

The first issue should be mitigated by a bigger architecture and more training 
data, since at the moment we are using only 2000 runs as before, which might 
not be enough given the higher complexity of the task.

The reason for the second problem is that the omniscient controller never 
overshoots the goal, so the network does not see this situation in training.
In Task 1, we solved the same issue with a special case: we had the omniscient 
controller move in reverse when it spawns inside the arms of the object. A 
similar solution could be applied here, extending this trick to arbitrary goal 
poses.

As a future work, we could try to see what happens to the current model when 
slightly changing the shape of the object. Moreover, we could also generalised 
the network by creating multiple polychromatic object with the faces coloured 
randomly. 

Another slightly more complex task could include adding obstacles that the 
robot should avoid, for example by orbiting around them.

In another spin-off, we can consider to add a second marXbot and learn to 
control the two robots with respect to each other (distributed control).

\section*{Acknowledgements}


\bibliographystyle{IEEEtran}
\bibliography{IEEEabrv, biblio}

\end{document}
